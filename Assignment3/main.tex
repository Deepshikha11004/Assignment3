\let\negmedspace\undefined
\let\negthickspace\undefined
\documentclass[journal, 12pt, twocolumn]{IEEEtran}
\usepackage{mathtools}
\usepackage{enumitem}
\usepackage{enumerate}
\usepackage{mathptmx}
\usepackage{amssymb}
\usepackage{textcomp}
\usepackage{mathrsfs}
\usepackage[euler]{textgreek}
\begin{document}
\title{ASSIGNMENT 2}
\author{Deepshikha(CS21BTECH11016)}
\maketitle

\begin{abstract}
This document contains the solution for Assignment 3 (Class 9 Maths NCERT Example 10).
\end{abstract}

\textbf{Example 10:}
5 people were asked about the time in a week they spend in doing
social work in their community. They said 10, 7, 13, 20 and 15 hours, respectively.
Find the mean (or average) time in a week devoted by them for social work.

\textbf{Solution:}
We have already studied in our earlier classes that the mean of a certain
number of observations is equal to\[\frac{\text{Sum of all the observations}}{\text{Total number of observations}}\]

To simplify our working of finding the mean, let us use a variable $x_i$
to denote the ith observation. In
this case, \emph{i} can take the values from 1 to 5. So our first observation is $x_1$, second
observation is $x_2$,and so on till $x_5$.

Also $x_1$ = 10 means that the value of the first observation, denoted by $x_1$,is 10.
Similarly, $x_2$ = 7, $x_3$ = 13, $x_4$ = 20 and $x_5$ = 15.

Therefore, the mean 
\begin{align}
\overline{x}&=\frac{\text{Sum of all the observations}}{\text{Total number of observations}}\\
    &=\frac{x_1 + x_2 + x_3 +x_4 +x_5}{5}\\
    &=\frac{10 + 7 + 13 + 20 + 15}{5}\\
    &=\frac{65}{3}\\
    &=13
\end{align}

So, the mean time spent by these 5 people in doing social work is 13 hours in a week.


Now, in case we are finding the mean time spent by 30 people in doing social
work, writing $x_1$ + $x_2$ + $x_3$ +.....+ $x_{30}$ would be a tedious job.We use the Greek symbol \textSigma (for the letter Sigma) for \emph{summation}.
Instead of writing $x_1$ + $x_2$ + $x_3$ +.....+ $x_{30}$, 
we write $\sum\limits_{i=1}^{30} x_i $ , which is read as ‘the sum of $x_i$ as \emph{i} varies from 1 to 30’.


So,
\begin{equation}
    \overline{x}=\frac{\sum\limits_{i=1}^{30} x_i}{30}\label{eq:6}
\end{equation}
Similarly, for n observations ,
\begin{equation}
    \overline{x}=\frac{\sum\limits_{i=1}^{n} x_i}{n}
\end{equation}


\end{document}